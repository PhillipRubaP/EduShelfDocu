\documentclass[twoside]{article}
\usepackage{graphicx} % Required for inserting images
\usepackage{amsmath}
\usepackage{fancyhdr}
\usepackage{listings}
\usepackage{xcolor}
\usepackage{tikz}
\usepackage{amssymb}
\usepackage{booktabs}
\usepackage{natbib}
%\usepackage{svg}
\newcommand{\includesvg}[2][]{\fbox{MISSING SVG: #2}}
% Load hyperref LAST to ensure it modifies citation commands properly
%\usepackage[colorlinks=true,linkcolor=blue,citecolor=blue,urlcolor=blue]{hyperref}

\usepackage{longtable}

\usepackage[backend=biber, style=numeric]{biblatex}
\usepackage[hidelinks, url=true]{hyperref}
\addbibresource{literatur.bib}

%\bibliographystyle{apalike}
%\bibliography{literatur} 

%\usepackage[numbers]{natbib}

%\usepackage[colorlinks=true,linkcolor=blue,citecolor=blue,urlcolor=blue]{hyperref}


\setlength{\footskip}{2cm} % Erhöhe den Abstand zwischen Text und Fußzeile
\renewcommand{\footrulewidth}{0.2pt} % Optional: Linie oberhalb der Fußzeile


\usetikzlibrary{shapes.geometric, arrows}

\tikzstyle{startstop} = [rectangle, rounded corners, minimum width=3cm, minimum height=1cm,text centered, draw=black, fill=red!30]
\tikzstyle{process} = [rectangle, minimum width=3cm, minimum height=1cm, text centered, draw=black, fill=blue!20]
\tikzstyle{arrow} = [thick,->,>=stealth]

\lstset{
    language=Python,
    backgroundcolor=\color{white},
    basicstyle=\ttfamily\small,
    keywordstyle=\color{blue},
    commentstyle=\color{green!40!black},
    stringstyle=\color{red},
    numbers=left,
    numberstyle=\tiny\color{gray},
    numbersep=5pt,
    breaklines=true,
    showstringspaces=false,
    frame=single,
    tabsize=4
}

\lstset{
    language=Swift,
    backgroundcolor=\color{white},
    basicstyle=\ttfamily\footnotesize,
    keywordstyle=\color{blue},
    commentstyle=\color{green!40!black},
    stringstyle=\color{red},
    numbers=left,
    numberstyle=\tiny\color{gray},
    numbersep=5pt,
    breaklines=true,
    showstringspaces=false,
    frame=single,
    tabsize=4
}

\lstset{captionpos=b}

%%%%%%%%%%%%%%%%%%%%%%%%%%%%%%%%%%%%%%%%%%%%%%%%%%%%%%%%%%%%%%%%%%%%%%%%%%%%%%%%%%%%%%%%%%%%%%%%%%%%%%%%%%%%%%%%%%%%%%%%%%%%%%%
%%     LaTeX Dokumentvorlage
%%     für Diplomarbeiten an der HTL Hollabrunn
%%     ---------------------------------------------------------------------
%%     für Verwendung mit
%%     - biblatex und biber
%%     - UTF8
%%     - TeX-Studio
%%     - siunitx - Paket
%%
%%     Installation unter Windows: erst MiKTeX und *dann* TeX-Studio installieren, fertig
%%%%%%%%%%%%%%%%%%%%%%%%%%%%%%%%%%%%%%%%%%%%%%%%%%%%%%%%%%%%%%%%%%%%%%%%%%%%%%%%%%%%%%%%%%%%%%%%%%%%%%%%%%%%%%%%%%%%%%%%%%%%%%%

\input{praeambel.tex}		% NICHT löschen
%%%%%%%%%%%%%%%%%%%%%%%%%%%%%%%%%%%%%%%%%%%%%%%%%%%
%%                   anpassen....                %%
%%%%%%%%%%%%%%%%%%%%%%%%%%%%%%%%%%%%%%%%%%%%%%%%%%%
%---------------------------------------- GRAPHIKPFAD (***anpassen***)  -----------------------------------------
% gleiches Verzeichnis oder Unterordner Bilder, auch absolute Pfade, NIE Backslash, CASE-sensitive!!!!
\graphicspath
{
	{./}
	{./img/}
}
\def\schuelerA{Phillip Rubak}
\def\schuelerB{Elias Welt}				% nicht weglöschen sondern \def\schuelerB{~}
\def\schuelerC{}           % nicht weglöschen sondern \def\schuelerC{~}
\def\schuelerD{}          % nicht weglöschen sondern \def\schuelerD{}

\newcommand{\schuelerAlle}{Phillip Rubak, Elias Welt}

\newcommand{\betreuerA}{Dipl.-Ing. Gerald Stoll}
\newcommand{\betreuerB}{}       % nicht weglöschen sondern zB \newcommand{\betreuerB}{}

\newcommand{\schuljahr}{2025/26}
\newcommand{\daTitel}{EduShelf}
\newcommand{\daTitelKurz}{EduShelf}
\newcommand{\datumErklaerung}{07.04.2025}			% Datum für die eidesstattliche Erklärung
\newcommand{\datumAbgabe}{07.04.2025}
\newcommand{\aktuellesSchuljahr}{2025/26}
\newcommand{\firma}{"X-Works"}


% Ab hier eigene Commands 

\newcommand{\nurSuad}{Suad Demiri}
\newcommand{\nurStefan}{Stefan Bayer}
\newcommand{\nurKurt}{Kurt Broneder}
\newcommand{\nurLorenz}{Lorenz Geyser}




\newcommand{\suadSeite}{
  \fancyfoot[C]{\nurSuad}
}

\newcommand{\stefanSeite}{
  \fancyfoot[C]{\nurStefan}
}

\newcommand{\kurtSeite}{
  \fancyfoot[C]{\nurKurt}
}

\newcommand{\lorenzSeite}{
  \fancyfoot[C]{\nurLorenz}
}



% Befehl zum Zurückschalten auf alle Autoren
\newcommand{\gemeinsamSeite}{
  \fancyfoot{\schuelerAlle}
}
	% Enthält alle Definiotinen wie Schüler und Betreuer
%%%%%%%%%%%%%%%%%%%%%%%%%%%%%%%%%%%%%%%%%%%%%%%%%%%%%%%%%%%%%%%%%%%%%%%%%%%%%%%%%%%%%%%%%%%%%%%%%%%%%%%%%%%%%%%%%%%%%%%%%%%%%%%
%%   unbedingt in TeX-Studio Optionen->TeX-Studio konfigurieren->Erzeugen->Standardbibliographieprogramm "Biber" einstellen  %%
%%%%%%%%%%%%%%%%%%%%%%%%%%%%%%%%%%%%%%%%%%%%%%%%%%%%%%%%%%%%%%%%%%%%%%%%%%%%%%%%%%%%%%%%%%%%%%%%%%%%%%%%%%%%%%%%%%%%%%%%%%%%%%%

%%%%%%%%%%%%%%%%%%%%%%%%%%%%%%%%%%%%%%%%%%%%%%%%%%%%%%%%%%%%%%%%%%%%%%%%%%%%%%%%%%%%%%%%%%%%%%%%%%%%%%%%%%%%%%%%%%%%%%%%%%%%%%%
%%                    falsche / unglückliche Abteilungen kann man hier für das ganze Dokument verändern                      %%
%%                    zB  Di-plomarbeit oder Diplomar-beit statt Diplom-arbeit                                               %%
%%%%%%%%%%%%%%%%%%%%%%%%%%%%%%%%%%%%%%%%%%%%%%%%%%%%%%%%%%%%%%%%%%%%%%%%%%%%%%%%%%%%%%%%%%%%%%%%%%%%%%%%%%%%%%%%%%%%%%%%%%%%%%%
\hyphenation{Diplom-arbeit}
\hyphenation{En-ti-ty-Stu-dent-Has-Assign-ment}


%%%%%%%%%%%%%%%%%%%%%%%%%%%%%%%%%%%%%%%%%%%%%%%%%%%%%%%%%%%%%%%%%%%%%%%%%%%%%%%%%%%%%%%%%%%%%%%%%%%%%%%%%%%%%%%%%%%%%%%%%%%%%%%
%                                         an ihre Teil-Dokumente anpassen                                                    %%
%%%%%%%%%%%%%%%%%%%%%%%%%%%%%%%%%%%%%%%%%%%%%%%%%%%%%%%%%%%%%%%%%%%%%%%%%%%%%%%%%%%%%%%%%%%%%%%%%%%%%%%%%%%%%%%%%%%%%%%%%%%%%%%
\begin{document}
	
	\input{listings_setup.tex}  			% Einbinden von Quellcode, muss für andere Sprachen als bash, muss angepasst werden
	
	%%%%%%%%%%%%%%%%%%%%%%%%%%%%%%%%%%%%%%%%%%%%%%%%%%%
    %%            hier nichts ändern                 %%
    %%%%%%%%%%%%%%%%%%%%%%%%%%%%%%%%%%%%%%%%%%%%%%%%%%%	
	
	\newgeometry{top=2cm,left=2cm,right=2cm}
	\begin{titlepage}
		
		\begin{center}
			\begin{tabular}{ c c c }
				\thead{\includegraphics[width=33mm]{img/htlhl_bildung_mit_zukunft}} \hspace{15mm} &
				\thead{\Large{\textbf{HTBL Hollabrunn}} \\ \small{\textbf{Höhere Lehranstalt für Informationstechnologie}}} &
                \hspace{15mm}
				\thead{\includegraphics[width=21mm]{img/htlhl_logo}}
			\end{tabular}

            \vspace{15mm}
			
			\Huge{\textbf{Diplomarbeit}}
		    
		    \vspace{15mm}
		    
			\Huge{\textbf{\daTitel}}	 
		\end{center}
		
		\vspace{20mm}	% Hier könnte eine Abbildung gut aussehen
		
		\begin{large}
			\begin{tabular}{p{25mm}p{100mm}}
			 Verfasser: &  Phillip Rubak 5AHITS	\\
			            &  Elias Welt 5AHITS	\\
			 ~			& 				\\
			 Betreuer:  & \betreuerA	\\
			\end{tabular} 
		\end{large}
		
		\vspace{61mm}
		
		\begin{large}
			Schuljahr: \aktuellesSchuljahr
			~ \\
			~ \\
			\rule{\textwidth}{1.0pt} \\
			~ \\
			Abgabevermerk: \\
			~ \\
			~ \\
			Datum: \datumAbgabe \qquad \qquad \qquad übernommen von:
		\end{large}
		
	\end{titlepage}
	\newpage\null\thispagestyle{empty}\newpage % Irgendwie ganz leere Seite, ka
	\input{erklaerung.tex}
	\newpage\null\thispagestyle{empty}\newpage
	\chapter{HINWEISE}
\markright{Hinweise}
\\
\\
Die vorliegende Diplomarbeit wurde in Zusammenarbeit mit der Firma \firma \ ausgeführt.
\\
Die in dieser Diplomarbeit entwickelten Prototypen und Software-Produkte dürfen ganz oder auch in Teilen von Privatpersonen oder Firmen nur dann in Verkehr gebracht werden, wenn sie diese selbst geprüft und für den vorgesehenen Verwendungszweck für geeignet befunden haben.
Es wird keinerlei Haftung übernommen für irgendwelche Schäden, die aus der Nutzung der hier entwickelten oder beschriebenen Bestandteile des Projekts resultieren.
\\
\\
Für alle Entwicklungen gilt die GNU General Public License [http://www.gnu.org/licenses/gpl.html] der Free Software Foundation, Boston, USA in der Version 3.
~ \\


	\chapter{SCHLÜSSELBEGRIFFE}
\markright{Schlüsselbegriffe}
\begin{itemize}
    \item \texttt{TeamTrackr}
    \item \texttt{Android - Entwicklung}
    \item \texttt{iOS - Entwicklung}
    \item \texttt{Backend}
    \item \texttt{Frontend}
    \item \texttt{AI}    
\end{itemize}
\\

    \vspace{2\baselineskip}
	\chapter{DANKSAGUNGEN}
\markright{Danksagungen}
\\
\\
Wir möchten an dieser Stelle unseren aufrichtigen Dank all jenen aussprechen, die uns während der Entstehung dieser Diplomarbeit unterstützt haben.

Unser besonderer Dank gilt Tailored Apps, unserer Partnerfirma, die uns nicht nur mit wertvollen Einblicken und praxisnaher Unterstützung begleitet hat, sondern auch eine entscheidende Rolle in der erfolgreichen Umsetzung unseres Projekts gespielt hat. Die Zusammenarbeit mit euch war für uns eine wertvolle Erfahrung.

Ebenso möchten wir uns herzlich bei Herrn Geischläger, unserem Diplomarbeitsbetreuer, bedanken. Seine fachliche Expertise, seine konstruktiven Anregungen und seine stetige Unterstützung haben maßgeblich zur Qualität unserer Arbeit beigetragen.

Ein großer Dank gilt auch allen, die uns auf diesem Weg unterstützt haben – sei es durch hilfreiches Feedback, motivierende Worte oder ihr Verständnis und ihre Geduld.

Ohne diese wertvolle Unterstützung wäre unsere Diplomarbeit in dieser Form nicht möglich gewesen.
\newpage
	\markright{}
	% Kurzfassung
\begin{center}
	\begin{table}[h!]
		\begin{tabular}{|p{50mm}|p{110mm}|}
			\hline
			\thead{\includegraphics[width=33mm]{htlhl_bildung_mit_zukunft}} &
			\thead{{\textbf{HÖHERE TECHNISCHE BUNDESLEHRANSTALT HOLLABRUNN}} \\\hline \\ \Large{\textbf{Informationstechnologie}}}\\
			\hline
		\end{tabular}
	\end{table}
	~ \\
	~ \\
	\Huge{\textbf{DIPLOMARBEIT}}\\
	\LARGE{\textbf{DOKUMENTATION}}
	~ \\
	~ \\
	\begin{table}[h!]
		\begin{tabular}{|p{50mm}|p{110mm}|}
			\hline
			\thead{Name der \\ Verfasser/innen} &
			\thead{\schuelerA, \schuelerB} \\
			\hline
			\thead{Jahrgang \\ Schuljahr} &
			\thead{\aktuellesSchuljahr} \\
			\hline
			\thead{Thema der Diplomarbeit} &
			\thead{\daTitel} \\
			\hline
			\thead{Kooperationspartner} &
			\thead{-------} \\
			\hline
		\end{tabular}
	\end{table}
	\begin{table}[h!]
    \centering
    \begin{tabular}{|p{50mm}|p{110mm}|}
        \hline
        \makecell[l]{\small Aufgabenstellung} &
        \small{Im Rahmen dieser Diplomarbeit wird die Entwicklung einer innovativen App für Teamkommunikation und Aufgabenmanagement vorgestellt. 
        Ziel ist es, eine nahtlose Lösung bereitzustellen, die Organisationen und Teams dabei unterstützt, effizient zusammenzuarbeiten und Projekte erfolgreich zu managen. 
        Die App integriert essenzielle Funktionen wie Authentifizierung, Kalender, Chat inklusive KI-gestütztem Chatbot sowie die Erstellung und Verwaltung von Teams. 
        Darüber hinaus ermöglicht sie ein umfassendes Budget- und Zeitmanagement auf Projektebene.

        Ein zentraler Bestandteil des Projekts ist die Entwicklung nativer Anwendungen für iOS, Android, Desktop und Web sowie eines leistungsfähigen Backends. 
        Das Backend gewährleistet die Verwaltung aller Frontend-Anforderungen, speichert Daten in einer zentralen Datenbank und verarbeitet Benutzeranfragen effizient. 
        Durch die Integration aller Funktionen in einer benutzerfreundlichen Plattform wird die Fragmentierung bestehender Tools überwunden und eine reibungslose Zusammenarbeit gefördert.

        Das geplante Endprodukt stellt eine umfassende Lösung für modernes Projektmanagement dar, die die Bedürfnisse moderner Teams adressiert und die Effizienz sowie den Erfolg in Organisationen steigert.} \\
        \hline
    \end{tabular}
\end{table}
	\begin{table}[h!]
		\begin{tabular}{|p{50mm}|p{110mm}|}
			\hline
			\makecell[l{{p{50mm}}}]{\small{Realisierung}} &
			\makecell*[{{p{110mm}}}]{\small{Die Anliegen der Themenstellungen umfassen die Entwicklung einer iOS / Android / Desktop / Web App sowie eines Backends. Die nativen Applikationen sollen folgende Kernfeatures beinhalten Login und Registrierung (Authentifizierung), Kalender, eine Chat-Möglichkeit inklusive AI unterstützten Chatbot, sowie die Möglichkeit zur Erstellung und Verwaltung von Teams, in diesen soll es die Opportunität zur Budgetverwaltung und Zeitverwaltung von einzelnen Projekten geben. Die Webapplikation soll die selben bzw. ähnlich implementierten Features haben. Das Backend soll in der Lage sein alle Anforderungen der jeweiligen Frontends zu verwalten sowie in der Datenbank zu speichern und deren Abrufe zu verarbeiten.}} \\
			\hline
		\end{tabular}
	\end{table}
	\begin{table}[h!]
		\begin{tabular}{|p{50mm}|p{110mm}|}
			\hline
			\makecell[l{{p{50mm}}}]{\small{Ergebnisse}} &
			\makecell*[{{p{110mm}}}]{\small{Das geplante Ergebnis umfasst die fertigen iOS-, Android-, Desktop- und Web-Anwendungen sowie ein Backend. Die nativen Applikationen beinhalten folgende Kernfunktionen: Benutzer-Login und -Registrierung (Authentifizierung), einen Kalender, eine Chat-Funktion mit einem KI-gestützten Chatbot sowie die Möglichkeit zur Erstellung und Verwaltung von Teams und deren Projekten. Innerhalb der Teams werden Funktionen für Budget- und Zeitmanagement bereitgestellt. Die Webanwendung hat ähnliche oder identische Funktionen. Das Backend ist so entwickelt, dass es die Anforderungen der jeweiligen Frontends erfüllt, Daten in einer Datenbank speichert und diese für verschiedene Anfragen der Applikationen verfügbar macht. Des Weiteren gewährleisten die Anwendungen eine benutzerfreundliche Bedienung.}} \\
			\hline
		\end{tabular}
	\end{table}

	\newpage

	\begin{table}[h!]
		\begin{tabular}{|p{50mm}|p{110mm}|}
    \hline
    \makecell[l{{p{50mm}}}]{\small{Typische Grafik, Foto etc. (mit Erläuterung)}} &
    \makecell*[{{p{110mm}}}]{\vspace{12mm} \includesvg[width=88mm]{img/blockschaltbildSVG.svg} \vspace{10mm}} \\ % No file extension here!
    \hline
\end{tabular}
	\end{table}
	\begin{table}[h!]
		\begin{tabular}{|p{50mm}|p{110mm}|}
			\hline
			\makecell[l{{p{50mm}}}]{\small{Teilnahme an Wettbewerben, Auszeichnungen}} &
			\makecell*[{{p{110mm}}}]{\small{}} \\
			\hline
		\end{tabular}
	\end{table}
	\begin{table}[h!]
		\begin{tabular}{|p{50mm}|p{110mm}|}
			\hline
			\makecell[l{{p{50mm}}}]{
				\small{Möglichkeiten der}\\
				\small{Einsichtnahme in die Arbeit}} &
			\makecell*[{{p{110mm}}}]{
				\small{HTL Hollabrunn} \\
				\small{Anton Ehrenfriedstraße 10} \\
				\small{2020 Hollabrunn}} \\
			\hline
		\end{tabular}
	\end{table}
    \vspace{8mm}
	\begin{table}[h!]
		\begin{tabular}{|p{52mm}|p{52mm}|p{52mm}|}
			\hline
			\makecell[l{{p{52mm}}}]{
				\small{Approbation} \\
				\small{(Datum / Unterschrift)} \\ ~ \\ ~ \\ ~ \\} &
			\makecell[l{{p{52mm}}}]{\small{Prüfer/Prüferin} \\ ~ \\ ~ \\ ~ \\ ~ \\} &
			\makecell[l{{p{52mm}}}]{
				\footnotesize{Direktor/Direktorin} \\
				\footnotesize{Abteilungsvorstand/} \\
				\footnotesize{Abteilungsvorständin}\\ ~ \\ ~ \\} \\
			\hline
		\end{tabular}
	\end{table}
\end{center}

\newpage
	% Abstract
\begin{center}
	\begin{table}[h!]
		\begin{tabular}{|p{50mm}|p{110mm}|}
			\hline
			\thead{\includegraphics[width=33mm]{htlhl_bildung_mit_zukunft}} &
			\thead{
				\textbf{HÖHERE TECHNISCHE BUNDESLEHRANSTALT HOLLABRUNN} \\
				\textbf{COLLEGE of ENGINEERING } \\
				\hline \\
				\Large{\textbf{Information Technology}}} \\
			\hline
		\end{tabular}
	\end{table}
	~ \\
	~ \\
	\Huge{\textbf{DIPLOMA THESIS}}\\
	\LARGE{\textbf{Documentation}}
	~ \\
	~ \\
	\begin{table}[h!]
		\begin{tabular}{|p{50mm}|p{110mm}|}
			\hline
			\thead{Author(s)} &
			\thead{\schuelerA, \schuelerB, \\ \schuelerC, \schuelerD} \\
			\hline
			\thead{From \\ Academic year} &
			\thead{\aktuellesSchuljahr} \\
			\hline
			\thead{Topic} &
			\thead{\daTitel} \\
			\hline
			\thead{Co-operation partners} &
			\thead{TailoredApps} \\
			\hline
		\end{tabular}
	\end{table}
	\begin{table}[h!]
    \centering
    \begin{tabular}{|p{50mm}|p{110mm}|}
        \hline
        \makecell[l]{\small Task Description} &
        \small{This diploma project focuses on developing an innovative app for team communication and task management. 
        The goal is to provide a seamless solution that helps organizations and teams collaborate efficiently and successfully manage projects. 
        The app integrates essential features such as authentication, a calendar, chat including an AI-powered chatbot, and team creation and management. 
        Additionally, it offers comprehensive budget and time management at the project level.

        A key aspect of the project is the development of native applications for iOS, Android, Desktop, and Web, along with a powerful backend. 
        The backend handles all frontend requests, stores data in a central database, and processes user interactions efficiently. 
        By integrating all functions into a user-friendly platform, the fragmentation of existing tools is reduced, fostering smooth collaboration.

        The final product will provide a comprehensive solution for modern project management, addressing the needs of contemporary teams while enhancing efficiency and success within organizations.} \\
        \hline
    \end{tabular}
\end{table}

\begin{table}[h!]
    \centering
    \begin{tabular}{|p{50mm}|p{110mm}|}
        \hline
        \makecell[l]{\small Implementation} &
        \small{The project involves developing a mobile and desktop application for iOS, Android, Desktop, and Web, as well as a backend system. 
        The native applications will include core features such as login and registration (authentication), a calendar, chat functionality with AI support, and the ability to create and manage teams. 
        Within these teams, users will have options for budget and time management at the project level. 
        The web application will offer similar features. 
        The backend will be designed to manage all frontend requirements, store data in a database, and process requests efficiently.} \\
        \hline
    \end{tabular}
\end{table}

\begin{table}[h!]
    \centering
    \begin{tabular}{|p{50mm}|p{110mm}|}
        \hline
        \makecell[l]{\small Results} &
        \small{The expected outcome includes fully developed iOS, Android, Desktop, and Web applications, along with a backend system. 
        The native applications will feature user login and registration (authentication), a calendar, chat functionality with an AI-powered chatbot, and the ability to create and manage teams and projects. 
        Teams will have access to budget and time management tools. 
        The web application will provide similar or identical functionality. 
        The backend will be designed to support all frontend applications, store data in a database, and efficiently process requests. 
        Additionally, the applications will ensure a user-friendly experience.} \\
        \hline
    \end{tabular}
\end{table}


	\newpage

    \newpage



\iffalse

% Abstract text insertion
\begin{abstract}
    \section*{Kurzfassung (Deutsch)}
    Im Rahmen dieser Diplomarbeit wird die Entwicklung einer innovativen App für Teamkommunikation und Aufgabenmanagement vorgestellt. Ziel ist es, eine nahtlose Lösung bereitzustellen, die Organisationen und Teams dabei unterstützt, effizient zusammenzuarbeiten und Projekte erfolgreich zu managen. Die App integriert essenzielle Funktionen wie Authentifizierung, Kalender, Chat inklusive KI-gestütztem Chatbot sowie die Erstellung und Verwaltung von Teams. Darüber hinaus ermöglicht sie ein umfassendes Budget- und Zeitmanagement auf Projektebene.

    Ein zentraler Bestandteil des Projekts ist die Entwicklung nativer Anwendungen für iOS, Android, Desktop und Web sowie eines leistungsfähigen Backends. Das Backend gewährleistet die Verwaltung aller Frontend-Anforderungen, speichert Daten in einer zentralen Datenbank und verarbeitet Benutzeranfragen effizient. Durch die Integration aller Funktionen in einer benutzerfreundlichen Plattform wird die Fragmentierung bestehender Tools überwunden und eine reibungslose Zusammenarbeit gefördert.

    Das geplante Endprodukt stellt eine umfassende Lösung für modernes Projektmanagement dar, die die Bedürfnisse moderner Teams adressiert und die Effizienz sowie den Erfolg in Organisationen steigert.

    \section*{Abstract (English)}
    This thesis presents the development of an innovative app for team communication and task management. The goal is to provide a seamless solution that helps organizations and teams collaborate efficiently and manage projects successfully. The app integrates essential features such as authentication, a calendar, a chat function with an AI-powered chatbot, and tools for creating and managing teams. Additionally, it enables comprehensive budget and time management at the project level.

    A key component of the project is the development of native applications for iOS, Android, Desktop, and Web, alongside a robust backend. The backend ensures the management of all frontend requirements, stores data in a central database, and efficiently processes user requests. By integrating all functionalities into a user-friendly platform, the solution overcomes the fragmentation of existing tools and promotes seamless collaboration.

    The final product is designed to be a comprehensive solution for modern project management, addressing the needs of contemporary teams and enhancing efficiency and success within organizations.
\end{abstract}
\fi

\newpage

\begin{table}[h!]
    \centering
    \begin{tabular}{|p{50mm}|p{110mm}|}
    \hline
    \makecell[l{{p{50mm}}}]{\small{Typische Grafik, Foto etc. (mit Erläuterung)}} &
    \makecell*[{{p{110mm}}}]{\vspace{12mm} \includesvg[width=88mm]{img/blockschaltbildSVG.svg} \vspace{10mm}} \\ % No file extension here!
    \hline
\end{tabular}
\end{table}
	\begin{table}[h!]
		\begin{tabular}{|p{50mm}|p{110mm}|}
			\hline
			\makecell[l{{p{50mm}}}]{\small{Participation in competitions Awards}} &
			\makecell*[{{p{110mm}}}]{\small{}} \\
			\hline
		\end{tabular}
	\end{table}
	\begin{table}[h!]
		\begin{tabular}{|p{50mm}|p{110mm}|}
			\hline
			\makecell[l{{p{50mm}}}]{
				\small{Accessibility of}\\
				\small{final project thesis}} &
			\makecell*[{{p{110mm}}}]{
				\small{HTL Hollabrunn} \\
				\small{Anton Ehrenfriedstraße 10} \\
				\small{2020 Hollabrunn}} \\
			\hline
		\end{tabular}
	\end{table}
    \vspace{8mm}
	\begin{table}[h!]
		\begin{tabular}{|p{52mm}|p{52mm}|p{52mm}|}
			\hline
			\makecell[l{{p{52mm}}}]{\small{Approval (Date / Signature)} \\ ~ \\ ~ \\} &
			\makecell[l{{p{52mm}}}]{\small{Examiner/s} \\ ~ \\ ~ \\} &
			\makecell[l{{p{52mm}}}]{\small{Head of Department / College} \\ ~ \\ ~ \\} \\
			\hline
		\end{tabular}
	\end{table}
\end{center}

\newpage
    \section*{Thema}
TeamTrackr

\section*{Verlauf}
\begin{itemize}
    \item 13.06.2024: Das Thema 'TeamTrackr' wurde durch Franz Geischläger angelegt.
    \item 13.06.2024: Sie wurden zum Thema hinzugefügt.
    \item 13.06.2024: Sie wurden zum Thema hinzugefügt.
    \item 13.06.2024: Sie wurden zum Thema hinzugefügt.
    \item 16.09.2024: Thema wurde eingereicht.
    \item 16.09.2024: Thema wurde von Franz Geischläger akzeptiert.
    \item 16.09.2024: Thema wurde von Wilfried Trollmann akzeptiert.
    \item 23.09.2024: Thema wurde von Wolfgang Bodei akzeptiert.
\end{itemize}

\section*{Schulische Informationen}
\begin{itemize}
    \item \textbf{Schulart:} Technische, gewerbliche und kunstgewerbliche Schulen
    \item \textbf{Klasse:} 5AHITS
    \item \textbf{Abteilung:} IT
    \item \textbf{Schuljahr der abschließenden Prüfung:} 2024/25
\end{itemize}

\section*{Betreuung}
\begin{longtable}{ll}
    \toprule
    \textbf{Rolle} & \textbf{Name} \\
    \midrule
    Hauptverantwortlicher Betreuer & Franz Geischläger (Akzeptiert) \\
    Direktion & Wolfgang Bodei (Akzeptiert) \\
    Abteilungsvorstand & Wilfried Trollmann (Akzeptiert) \\
    \bottomrule
\end{longtable}
\newpage
\section*{Kooperationspartner / Auftraggeber}
\begin{itemize}
    \item \textbf{Name:} Tailored Apps
    \item \textbf{Adresse:} Heiligenstädterstraße 31/1, 1190 Wien
    \item \textbf{URL:} \texttt{www.tailored-apps.com}
    \item \textbf{E-Mail:} \texttt{office@tailored-apps.com}
    \item \textbf{Ansprechpartner:} Martin Zehetner
    \item \textbf{Telefon:} +43 18902845
    \item \textbf{Typ:} Sonstige
\end{itemize}

\section*{Beteiligte SchülerInnen}
\begin{longtable}{lll}
    \toprule
    \textbf{Schüler/in} & \textbf{Individuelle Themenstellung} & \textbf{Abteilung} \\
    \midrule
    Stefan Bayer & Android-Development & IT \\
    Kurt Broneder & Web-Development & IT \\
    Suad Demiri & iOS-Development & IT \\
    Lorenz Geyser & Backend-Development & IT \\
    \bottomrule
\end{longtable}



\section*{Ausgangslage}
Derzeit existiert keine Applikation, die es ermöglicht, Organisationen, Projekte und deren Teams effizient und benutzerfreundlich zu managen. Darüber hinaus fehlt eine Lösung, die alle essenziellen Tools integriert, um ein solches effizientes und benutzerfreundliches Management zu gewährleisten.

\section*{Untersuchungsanliegen der individuellen Themenstellungen}
Die Anliegen der Themenstellungen umfassen die Entwicklung einer iOS-, Android-, Desktop- und Web-App sowie eines Backends.  
Die nativen Applikationen sollen folgende Kernfeatures beinhalten:
\begin{itemize}
    \item Login und Registrierung (Authentifizierung)
    \item Kalender
    \item Eine Chat-Möglichkeit inklusive AI-unterstütztem Chatbot
    \item Erstellung und Verwaltung von Teams
    \item Budgetverwaltung und Zeitverwaltung von Projekten
\end{itemize}
Die Webapplikation soll dieselben bzw. ähnlich implementierten Features haben.  
Das Backend soll alle Anforderungen der jeweiligen Frontends verwalten, Daten in einer Datenbank speichern und Anfragen verarbeiten.

\section*{Zielsetzung}
Unsere App bietet eine nahtlose Lösung für Teamkommunikation und Aufgabenmanagement.  
Mit leistungsstarken Funktionen unterstützt sie Teams und Organisationen dabei, effizient zusammenzuarbeiten und ihre Projekte erfolgreich zu verfolgen.  
Durch die Integration von Kommunikation und Projektverfolgung löst unsere App das Problem der Fragmentierung und fördert eine reibungslose Zusammenarbeit.

\section*{Geplantes Ergebnis der individuellen Themenstellungen}
Das geplante Ergebnis umfasst die fertigen iOS-, Android-, Desktop- und Web-Anwendungen sowie ein Backend.  
Die nativen Applikationen beinhalten folgende Kernfunktionen:
\begin{itemize}
    \item Benutzer-Login und -Registrierung (Authentifizierung)
    \item Kalender
    \item Chat-Funktion mit einem KI-gestützten Chatbot
    \item Erstellung und Verwaltung von Teams und deren Projekten
    \item Budget- und Zeitmanagement innerhalb der Teams
\end{itemize}
Die Webanwendung wird ähnliche oder identische Funktionen haben.  
Das Backend wird so entwickelt, dass es die Anforderungen der jeweiligen Frontends erfüllt, Daten in einer Datenbank speichert und diese für verschiedene Anfragen der Applikationen verfügbar macht.  
Des Weiteren gewährleisten die Anwendungen eine benutzerfreundliche Bedienung.

\section*{Meilensteine}
\begin{longtable}{ll}
    \toprule
    \textbf{Meilenstein} & \textbf{Datum} \\
    \midrule
    Fertigstellung Grundkonzept & 20.06.2024 \\
    Fertigstellung Backend & 22.11.2024 \\
    Fertigstellung Frontend & 20.12.2024 \\
    Fertigstellung Applikationen & 24.01.2025 \\
    Fertigstellung Dokumentation & 14.03.2025 \\
    \bottomrule
\end{longtable}

\includepdf[pages=1]{ABA_Rechtliche_Erklärung-1.pdf}
    \pagestyle{alle}
	\newpage\null\thispagestyle{empty}\newpage
%	\includepdf[pages=-]{ABA_Erklaerung.pdf} 	% Anpassen

	\tableofcontents			% Erstellen des Inhaltsverzeichnisses
	\newpage
	
\section{Motivation für die Arbeit}
Die effektive Integration von Management-Instrumentarien für Organisationsstrukturen, Projektkoordination und Teamführung stellt ein signifikantes Desiderat in der aktuellen digitalen Infrastruktur dar. Gegenwärtige Analysen dokumentieren eine Fragmentierung der verfügbaren Softwarelösungen, wodurch die Realisierung eines kohärenten und effizienten Organisationsmanagements substantiell limitiert wird.  Die Diskrepanz zwischen vorhandenen partikulären Anwendungen und der Anforderung nach einer holistischen Managementplattform manifestiert sich in beträchtlichen Effizienzdefiziten sowie suboptimaler Ressourcenallokation innerhalb organisatorischer Einheiten. Diese Forschungsarbeit adressiert die identifizierte Problematik mittels einer systematischen Konzeption und Implementation einer integrierten Softwarelösung zur Überwindung existierender technologischer Limitationen.

\section{Umfeld und Stand der Technik}
Im Kontext der gegenwärtigen technologischen Entwicklung existieren divergente Einzelsysteme zur Adressierung spezifischer Management-Aspekte. Eine kritische Evaluation der Fachliteratur indiziert signifikante Korrelationen zwischen fragmentierten Management-Infrastrukturen und reduzierten Produktivitätsparametern sowie verminderter Nutzerakzeptanz. Die Extrapolation aktueller Entwicklungstendenzen im Bereich digitaler Management-Systeme lässt eine progressive Konvergenz einzelner Funktionalitäten antizipieren, jedoch fehlt bislang eine vollständig interoperable Plattform mit multimodaler Zugänglichkeit.

Existierende Systeme  offerieren elaborierte Funktionalitäten im Bereich der Kommunikationsfazilitation, während ein anderes System primär exzelliert im Kontext des Projektmanagements. Eine Integration dieser disparaten Ansätze in eine kohärente Systementität wurde bislang nicht realisiert. Die vorliegende Forschungsarbeit positioniert sich in dieser identifizierten Forschungslücke mit dem Ziel der Entwicklung einer konsolidierten Management-Infrastruktur.

\section{Aufgabenstellung}
Die Forschungsaufgabe umfasst die systematische Konzeption und Implementierung eines multimodalen Anwendungssystems mit nativen Applikationen für iOS, Android und Desktop-Systeme sowie einer komplementären Web-Applikation, unterstützt durch eine zentralisierte Backend-Infrastruktur. Die primären Forschungs- und Entwicklungsziele umfassen:

\begin{itemize}
    \item Entwicklung eines hochsicheren Authentifizierungsmechanismus unter Berücksichtigung aktueller kryptographischer Standards und Sicherheitsparadigmen
    \item Implementation eines semantisch strukturierten Kalendarsystems mit integrierter Ressourcenallokationsfunktionalität
    \item Konzeption und Realisierung eines Kommunikationssystems mit bidirektionaler Informationstransmission sowie Integration eines künstlich-intelligenten Konversationssystems zur Optimierung der Kommunikationseffizienz
\end{itemize}

Die Backend-Architektur erfordert eine optimierte Datenbankstruktur mit adäquater Skalierbarkeit sowie eine performante API-Implementation zur Gewährleistung effizienter plattformübergreifender Datensynchronisation und -konsistenz.

\subsection{Ziele}
Das primäre wissenschaftliche Ziel dieser Forschungsarbeit konstituiert sich in der Entwicklung und Evaluation einer vollständig funktionalen Management-Plattform mit multimodaler Zugänglichkeit auf diversen Endgeräten (iOS, Android, Desktop, Web) unter Berücksichtigung folgender Forschungsziele:

\begin{enumerate}
    \item Konzeption und Implementation einer optimierten Systemarchitektur mit effizienten Datenflussstrukturen zwischen Frontend- und Backend-Komponenten unter Anwendung aktueller Software-Engineering-Prinzipien
    
    \item Realisierung einer kognitiv ergonomischen Benutzeroberfläche mit konsistenter visueller und funktionaler Repräsentation über heterogene Plattformen hinweg, basierend auf etablierten Usability-Heuristiken
    
    \item Entwicklung eines kryptographisch robusten Authentifizierungssystems zur Gewährleistung der Datenintegrität und -vertraulichkeit
    
    \item Implementation eines multifunktionalen Team- und Projektmanagementsystems mit präzisen Funktionalitäten zur Budgetierung und temporalen Ressourcendisposition unter Berücksichtigung aktueller Projektmanagement-Methodologien
    
    \item Konzeption und Realisierung eines intelligenten Kommunikationssystems mit Integration von Natural Language Processing-Algorithmen zur Optimierung der Informationstransmission und Teamkoordination
\end{enumerate}

Die Verifizierung der Zielerreichung erfolgt mittels systematischer empirischer Evaluation unter Anwendung etablierter quantitativer und qualitativer Forschungsmethoden, um die Konformität mit den spezifizierten Anforderungen hinsichtlich Funktionalität, Benutzerfreundlichkeit und Systemperformanz zu validieren.
        \newpage


        
        \section{iOS}
        \suadSeite
\subsection{Einführung in Swift}

Swift ist eine moderne, leistungsstarke Programmiersprache, die von Apple Inc. im Jahr 2014 als Nachfolger von Objective-C entwickelt wurde. Sie wurde speziell für die Entwicklung von Anwendungen für iOS, macOS, watchOS und tvOS konzipiert und hat sich schnell als bevorzugte Sprache für die Apple-Plattform etabliert.

\subsubsection{Vorteile von Swift}

\begin{itemize}
  \item \textbf{Sicherheit:} Swift verfügt über ein strenges Typsystem und bietet umfassende Mechanismen zur Fehlervermeidung, wie optionale Typen und Wertprüfungen.
  \item \textbf{Leistung:} Die Sprache wurde auf Geschwindigkeit optimiert, welche mit C++ vergleichbar ist.
  \item \textbf{Lesbarkeit:} Swift hat eine klare, ausdrucksstarke Syntax, die leicht zu lesen und zu verstehen ist.
  \item \textbf{Wartbarkeit:} Durch moderne Sprachkonzepte wie automatische Speicherverwaltung, Typinferenz und funktionale Programmiermuster wird die Codewartung erleichtert.
  \item \textbf{Interoperabilität:} Swift kann mit Objective-C-Code koexistieren, was eine schrittweise Migration bestehender Projekte ermöglicht.
\end{itemize}

\subsubsection{Nachteile von Swift}

\begin{itemize}
  \item \textbf{Sprachevolution:} Die häufigen Updates der Sprache können zu Kompatibilitätsproblemen zwischen verschiedenen Swift-Versionen führen.
  \item \textbf{Lernkurve:} Trotz der klaren Syntax kann die Einarbeitung in fortgeschrittene Konzepte wie Protokollorientierte Programmierung anspruchsvoll sein.
  \item \textbf{Eingeschränktes Ökosystem:} Im Vergleich zu älteren Sprachen ist das Ökosystem von Swift, insbesondere für spezialisierte Bibliotheken, noch in der Entwicklung.
\end{itemize}






        \chapter*{Zeiterfassung}
\markright{Zeiterfassung}

\begin{figure}[H]
	\includegraphics[width=\textwidth]{img/Zeitaufwand_Demiri.png}
    \caption{AI und iOS Zeitaufwand}
\end{figure}
Gesamt: 263,17h\\
Freizeit: 228,48h
~ \\
\newpage


        \section{Frontend}
        \input{frontend/frontendMain.tex}
        \chapter*{Zeiterfassung}
\markright{Zeiterfassung}

\begin{figure}[H]
	\includegraphics[width=\textwidth]{img/Zeitaufwand_Demiri.png}
    \caption{AI und iOS Zeitaufwand}
\end{figure}
Gesamt: 263,17h\\
Freizeit: 228,48h
~ \\
\newpage


        \section{Backend}
        \input{backend/backendMain.tex}
        \chapter*{Zeiterfassung}
\markright{Zeiterfassung}

\begin{figure}[H]
	\includegraphics[width=\textwidth]{img/Zeitaufwand_Demiri.png}
    \caption{AI und iOS Zeitaufwand}
\end{figure}
Gesamt: 263,17h\\
Freizeit: 228,48h
~ \\
\newpage


        \pagestyle{alle}

        

        \pagestyle{alle}

        \printbibliography[heading=bibintoc, title={Quellenverzeichnis}]

        %\bibliographystyle{plainnat}
        %\bibliography{references}



        


        




        

	\listoffigures		% Abbildungen
	\lstlistoflistings	% Code
	%\printbibliography	% Literatur


\end{document}

