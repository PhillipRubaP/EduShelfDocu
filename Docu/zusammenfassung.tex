% Kurzfassung
\begin{center}
	\begin{table}[h!]
		\begin{tabular}{|p{50mm}|p{110mm}|}
			\hline
			\thead{\includegraphics[width=33mm]{htlhl_bildung_mit_zukunft}} &
			\thead{{\textbf{HÖHERE TECHNISCHE BUNDESLEHRANSTALT HOLLABRUNN}} \\\hline \\ \Large{\textbf{Informationstechnologie}}}\\
			\hline
		\end{tabular}
	\end{table}
	~ \\
	~ \\
	\Huge{\textbf{DIPLOMARBEIT}}\\
	\LARGE{\textbf{DOKUMENTATION}}
	~ \\
	~ \\
	\begin{table}[h!]
		\begin{tabular}{|p{50mm}|p{110mm}|}
			\hline
			\thead{Name der \\ Verfasser/innen} &
			\thead{\schuelerA, \schuelerB} \\
			\hline
			\thead{Jahrgang \\ Schuljahr} &
			\thead{\aktuellesSchuljahr} \\
			\hline
			\thead{Thema der Diplomarbeit} &
			\thead{\daTitel} \\
			\hline
			\thead{Kooperationspartner} &
			\thead{-------} \\
			\hline
		\end{tabular}
	\end{table}
	\begin{table}[h!]
    \centering
    \begin{tabular}{|p{50mm}|p{110mm}|}
        \hline
        \makecell[l]{\small Aufgabenstellung} &
        \small{Im Rahmen dieser Diplomarbeit wird die Entwicklung einer innovativen App für Teamkommunikation und Aufgabenmanagement vorgestellt. 
        Ziel ist es, eine nahtlose Lösung bereitzustellen, die Organisationen und Teams dabei unterstützt, effizient zusammenzuarbeiten und Projekte erfolgreich zu managen. 
        Die App integriert essenzielle Funktionen wie Authentifizierung, Kalender, Chat inklusive KI-gestütztem Chatbot sowie die Erstellung und Verwaltung von Teams. 
        Darüber hinaus ermöglicht sie ein umfassendes Budget- und Zeitmanagement auf Projektebene.

        Ein zentraler Bestandteil des Projekts ist die Entwicklung nativer Anwendungen für iOS, Android, Desktop und Web sowie eines leistungsfähigen Backends. 
        Das Backend gewährleistet die Verwaltung aller Frontend-Anforderungen, speichert Daten in einer zentralen Datenbank und verarbeitet Benutzeranfragen effizient. 
        Durch die Integration aller Funktionen in einer benutzerfreundlichen Plattform wird die Fragmentierung bestehender Tools überwunden und eine reibungslose Zusammenarbeit gefördert.

        Das geplante Endprodukt stellt eine umfassende Lösung für modernes Projektmanagement dar, die die Bedürfnisse moderner Teams adressiert und die Effizienz sowie den Erfolg in Organisationen steigert.} \\
        \hline
    \end{tabular}
\end{table}
	\begin{table}[h!]
		\begin{tabular}{|p{50mm}|p{110mm}|}
			\hline
			\makecell[l{{p{50mm}}}]{\small{Realisierung}} &
			\makecell*[{{p{110mm}}}]{\small{Die Anliegen der Themenstellungen umfassen die Entwicklung einer iOS / Android / Desktop / Web App sowie eines Backends. Die nativen Applikationen sollen folgende Kernfeatures beinhalten Login und Registrierung (Authentifizierung), Kalender, eine Chat-Möglichkeit inklusive AI unterstützten Chatbot, sowie die Möglichkeit zur Erstellung und Verwaltung von Teams, in diesen soll es die Opportunität zur Budgetverwaltung und Zeitverwaltung von einzelnen Projekten geben. Die Webapplikation soll die selben bzw. ähnlich implementierten Features haben. Das Backend soll in der Lage sein alle Anforderungen der jeweiligen Frontends zu verwalten sowie in der Datenbank zu speichern und deren Abrufe zu verarbeiten.}} \\
			\hline
		\end{tabular}
	\end{table}
	\begin{table}[h!]
		\begin{tabular}{|p{50mm}|p{110mm}|}
			\hline
			\makecell[l{{p{50mm}}}]{\small{Ergebnisse}} &
			\makecell*[{{p{110mm}}}]{\small{Das geplante Ergebnis umfasst die fertigen iOS-, Android-, Desktop- und Web-Anwendungen sowie ein Backend. Die nativen Applikationen beinhalten folgende Kernfunktionen: Benutzer-Login und -Registrierung (Authentifizierung), einen Kalender, eine Chat-Funktion mit einem KI-gestützten Chatbot sowie die Möglichkeit zur Erstellung und Verwaltung von Teams und deren Projekten. Innerhalb der Teams werden Funktionen für Budget- und Zeitmanagement bereitgestellt. Die Webanwendung hat ähnliche oder identische Funktionen. Das Backend ist so entwickelt, dass es die Anforderungen der jeweiligen Frontends erfüllt, Daten in einer Datenbank speichert und diese für verschiedene Anfragen der Applikationen verfügbar macht. Des Weiteren gewährleisten die Anwendungen eine benutzerfreundliche Bedienung.}} \\
			\hline
		\end{tabular}
	\end{table}

	\newpage

	\begin{table}[h!]
		\begin{tabular}{|p{50mm}|p{110mm}|}
    \hline
    \makecell[l{{p{50mm}}}]{\small{Typische Grafik, Foto etc. (mit Erläuterung)}} &
    \makecell*[{{p{110mm}}}]{\vspace{12mm} \includesvg[width=88mm]{img/blockschaltbildSVG.svg} \vspace{10mm}} \\ % No file extension here!
    \hline
\end{tabular}
	\end{table}
	\begin{table}[h!]
		\begin{tabular}{|p{50mm}|p{110mm}|}
			\hline
			\makecell[l{{p{50mm}}}]{\small{Teilnahme an Wettbewerben, Auszeichnungen}} &
			\makecell*[{{p{110mm}}}]{\small{}} \\
			\hline
		\end{tabular}
	\end{table}
	\begin{table}[h!]
		\begin{tabular}{|p{50mm}|p{110mm}|}
			\hline
			\makecell[l{{p{50mm}}}]{
				\small{Möglichkeiten der}\\
				\small{Einsichtnahme in die Arbeit}} &
			\makecell*[{{p{110mm}}}]{
				\small{HTL Hollabrunn} \\
				\small{Anton Ehrenfriedstraße 10} \\
				\small{2020 Hollabrunn}} \\
			\hline
		\end{tabular}
	\end{table}
    \vspace{8mm}
	\begin{table}[h!]
		\begin{tabular}{|p{52mm}|p{52mm}|p{52mm}|}
			\hline
			\makecell[l{{p{52mm}}}]{
				\small{Approbation} \\
				\small{(Datum / Unterschrift)} \\ ~ \\ ~ \\ ~ \\} &
			\makecell[l{{p{52mm}}}]{\small{Prüfer/Prüferin} \\ ~ \\ ~ \\ ~ \\ ~ \\} &
			\makecell[l{{p{52mm}}}]{
				\footnotesize{Direktor/Direktorin} \\
				\footnotesize{Abteilungsvorstand/} \\
				\footnotesize{Abteilungsvorständin}\\ ~ \\ ~ \\} \\
			\hline
		\end{tabular}
	\end{table}
\end{center}

\newpage