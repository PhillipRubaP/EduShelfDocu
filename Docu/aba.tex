\section*{Thema}
TeamTrackr

\section*{Verlauf}
\begin{itemize}
    \item 13.06.2024: Das Thema 'TeamTrackr' wurde durch Franz Geischläger angelegt.
    \item 13.06.2024: Sie wurden zum Thema hinzugefügt.
    \item 13.06.2024: Sie wurden zum Thema hinzugefügt.
    \item 13.06.2024: Sie wurden zum Thema hinzugefügt.
    \item 16.09.2024: Thema wurde eingereicht.
    \item 16.09.2024: Thema wurde von Franz Geischläger akzeptiert.
    \item 16.09.2024: Thema wurde von Wilfried Trollmann akzeptiert.
    \item 23.09.2024: Thema wurde von Wolfgang Bodei akzeptiert.
\end{itemize}

\section*{Schulische Informationen}
\begin{itemize}
    \item \textbf{Schulart:} Technische, gewerbliche und kunstgewerbliche Schulen
    \item \textbf{Klasse:} 5AHITS
    \item \textbf{Abteilung:} IT
    \item \textbf{Schuljahr der abschließenden Prüfung:} 2024/25
\end{itemize}

\section*{Betreuung}
\begin{longtable}{ll}
    \toprule
    \textbf{Rolle} & \textbf{Name} \\
    \midrule
    Hauptverantwortlicher Betreuer & Franz Geischläger (Akzeptiert) \\
    Direktion & Wolfgang Bodei (Akzeptiert) \\
    Abteilungsvorstand & Wilfried Trollmann (Akzeptiert) \\
    \bottomrule
\end{longtable}
\newpage
\section*{Kooperationspartner / Auftraggeber}
\begin{itemize}
    \item \textbf{Name:} Tailored Apps
    \item \textbf{Adresse:} Heiligenstädterstraße 31/1, 1190 Wien
    \item \textbf{URL:} \texttt{www.tailored-apps.com}
    \item \textbf{E-Mail:} \texttt{office@tailored-apps.com}
    \item \textbf{Ansprechpartner:} Martin Zehetner
    \item \textbf{Telefon:} +43 18902845
    \item \textbf{Typ:} Sonstige
\end{itemize}

\section*{Beteiligte SchülerInnen}
\begin{longtable}{lll}
    \toprule
    \textbf{Schüler/in} & \textbf{Individuelle Themenstellung} & \textbf{Abteilung} \\
    \midrule
    Stefan Bayer & Android-Development & IT \\
    Kurt Broneder & Web-Development & IT \\
    Suad Demiri & iOS-Development & IT \\
    Lorenz Geyser & Backend-Development & IT \\
    \bottomrule
\end{longtable}



\section*{Ausgangslage}
Derzeit existiert keine Applikation, die es ermöglicht, Organisationen, Projekte und deren Teams effizient und benutzerfreundlich zu managen. Darüber hinaus fehlt eine Lösung, die alle essenziellen Tools integriert, um ein solches effizientes und benutzerfreundliches Management zu gewährleisten.

\section*{Untersuchungsanliegen der individuellen Themenstellungen}
Die Anliegen der Themenstellungen umfassen die Entwicklung einer iOS-, Android-, Desktop- und Web-App sowie eines Backends.  
Die nativen Applikationen sollen folgende Kernfeatures beinhalten:
\begin{itemize}
    \item Login und Registrierung (Authentifizierung)
    \item Kalender
    \item Eine Chat-Möglichkeit inklusive AI-unterstütztem Chatbot
    \item Erstellung und Verwaltung von Teams
    \item Budgetverwaltung und Zeitverwaltung von Projekten
\end{itemize}
Die Webapplikation soll dieselben bzw. ähnlich implementierten Features haben.  
Das Backend soll alle Anforderungen der jeweiligen Frontends verwalten, Daten in einer Datenbank speichern und Anfragen verarbeiten.

\section*{Zielsetzung}
Unsere App bietet eine nahtlose Lösung für Teamkommunikation und Aufgabenmanagement.  
Mit leistungsstarken Funktionen unterstützt sie Teams und Organisationen dabei, effizient zusammenzuarbeiten und ihre Projekte erfolgreich zu verfolgen.  
Durch die Integration von Kommunikation und Projektverfolgung löst unsere App das Problem der Fragmentierung und fördert eine reibungslose Zusammenarbeit.

\section*{Geplantes Ergebnis der individuellen Themenstellungen}
Das geplante Ergebnis umfasst die fertigen iOS-, Android-, Desktop- und Web-Anwendungen sowie ein Backend.  
Die nativen Applikationen beinhalten folgende Kernfunktionen:
\begin{itemize}
    \item Benutzer-Login und -Registrierung (Authentifizierung)
    \item Kalender
    \item Chat-Funktion mit einem KI-gestützten Chatbot
    \item Erstellung und Verwaltung von Teams und deren Projekten
    \item Budget- und Zeitmanagement innerhalb der Teams
\end{itemize}
Die Webanwendung wird ähnliche oder identische Funktionen haben.  
Das Backend wird so entwickelt, dass es die Anforderungen der jeweiligen Frontends erfüllt, Daten in einer Datenbank speichert und diese für verschiedene Anfragen der Applikationen verfügbar macht.  
Des Weiteren gewährleisten die Anwendungen eine benutzerfreundliche Bedienung.

\section*{Meilensteine}
\begin{longtable}{ll}
    \toprule
    \textbf{Meilenstein} & \textbf{Datum} \\
    \midrule
    Fertigstellung Grundkonzept & 20.06.2024 \\
    Fertigstellung Backend & 22.11.2024 \\
    Fertigstellung Frontend & 20.12.2024 \\
    Fertigstellung Applikationen & 24.01.2025 \\
    Fertigstellung Dokumentation & 14.03.2025 \\
    \bottomrule
\end{longtable}

\includepdf[pages=1]{ABA_Rechtliche_Erklärung-1.pdf}